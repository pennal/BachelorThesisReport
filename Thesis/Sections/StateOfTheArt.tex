\section{State of the art}\label{sec:stateOfTheArt}
In the past years there have been a few attempts at summarization for software engineering. Rastkar \textit{et al.} \cite{A4} base their approach on pre-existing techniques generally used to summarize the contents of email threads, and apply the same principles to generate summaries of bug reports. Lotufo \textit{et al.} \cite{A5} also summarize bug reports, but base their approach on PageRank \cite{ilprints422}. 

Mani \textit{et al.} \cite{A7} proposes an implementation based on different techniques (e.g. Grasshopper, DivRank, Centroid) to generate summaries in an unsupervised manner. All of these approaches only consider plain text artifacts, whereas information retrieval techniques have been used to summarize source code. Such techniques include Vector Space Model and Latent Semantic Indexing, which were used to summarize code samples \cite{A8} \cite{A9}. 

The above examples all generate summaries to reduce information overload. Other techniques include the usage of recommender systems for software engineering \cite{B3}, which suggest relevant artifacts to the developer and tend to harness different sources. The sources can include APIs as presented by Rigby \textit{et al.} \cite{B4} \cite{B5}, or by extracting information from Q\&A sites, as presented by \cite{B6}, or by analyzing existing code bases \cite{B7} \cite{B8} \cite{B9} \cite{B10}. 

Ponzanelli \textit{et al.} \cite{Ponz2017a} present LIBRA, a holistic recommender based on a meta-information system which is capable of dealing with the heterogeneous nature of resources, as well as considering the current context.
