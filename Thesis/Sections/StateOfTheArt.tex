\section{State of the art}\label{sec:stateOfTheArt}
\todo[inline]{Still a major TODO...}
\newpage
%\subsection{Recommender Systems}
%The idea of a recommender ties directly into one of a summariser. The correlation is easy to see: both approaches try to show the end user relevant information that has to do with the current context. We start by analysing some currently existing recommender systems.
%
%Recommender systems have existed for quite a while, outside of Software Engineering. Firms like Ebay, Amazon and many other online stores have been showing clients related and potentially interesting purchases. The data is usually computed using the previous purchases, related searches, and data from other clients that have looked at the same item. 
%
%Another example is streaming services, such as Netflix or Spotify. Both of these services show the user what they may like, based on the history of a particular user, trending content, newly released material, etc. 
%
%These suggestions are tailored to the user and tend to be accurate. We can deduce that these systems may be holistic in their implementation, being as they do not only consider the current item, but many other factors such as user history
%
%Looking at Software engineering, the principle is still the same: recommend to the developer relevant documentation, questions, examples that may help in solving the current problem. However, what these systems also bring is information overload. By showing the developer more and more information, she may be overwhelmed by the sheer amount of it, causing more confusion. This is where summarisation plays a key component.
%
%\subsection{Summarisation}
