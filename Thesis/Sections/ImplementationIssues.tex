\section{Future Work}\label{sec:implementationIssues}
\subsection{Granularity}
Although this approach to summarisation works, the biggest limitation is the granularity of it. As we have seen, the service is able to correctly identify the importance of each unit inside of a document or collection of documents, but what is still missing is a true summarisation of the content. More specifically, what would be interesting to achieve is a summarisation algorithm that is able to take a text, and recreate a new one by following grammar rules and keeping the context of such intact.

Although interesting, it would bring a new set of challenges, particularly when dealing with code that is embedded inside of a sentence, forcing it to be shown no matter what in the summary. Making the summariser understand exactly how to connect code and text may prove to be quite the challenge

\subsection{Persistence}
The current implementation of the web service uses an in memory store, meaning that there is no real database behind the service, and the data is kept in memory and destroyed once the service is shut down. Although not essential to the project, a database would be useful, allowing the user to retrieve data that is older than what is saved in current memory. 

This brings the challenge of deciding what has to be kept, what has to be discarded and when. By keeping all of the units stored in a database, the user may see information that is too old or has nothing to do with the current context. A new way to tag the information would be required, tagging what the context was, when the information unit was added, and whether to discard it once new units are added to the current graph.

\subsection{Performance}
To implement the algorithm behind \projectName, many libraries were used. These libraries tend to have their own way of doing things, which our code had to adapt to. One example is the \scala{SimilarityParameters} present in the StORMeD devkit. These parameters have to be recalculated every time a new unit is added, which requires a lot of time. If we were only to consider small graphs, this would be no problem as there is almost no impact on the general performance. Once we start to consider larger graphs, the limitations start to show, and the performance of the whole system suffers. In order to fix such issue, the devkit would have to be rewritten from scratch, which is outside the scope of this project. 
