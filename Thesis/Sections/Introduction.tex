\section{Introduction}\label{sec:introduction}
%To solve the task at hand developers often rely on the world wide web to find answers. Developers are then faced with documentation, Q\&A sites, as well as tutorials which they can consult. Although exhaustive documentation, well crafted tutorials, or \dots questions are usually preferred, there are situations where the amount of information is too large, and requires developers to sift through the material to find the most relevant secitons, possibly among multiple pages, to help solve the task at hand. This can lead to information overload, which can be mitigated by utilizing summarization. 
%
%In this project, we introduce \projectName, a tool which provides the ability to create interactive extractive summaries of a single page or set of pages.  By building the tool using HoliRank\cite{Ponz2017a}, an algorithm which builds on the foundations of PageRank\cite{ilprints422}, the heterogeneous nature of artefacts is considered. By providing a simple UI, the developer 


In software engineering, the development of new products can be broken down into multiple phases: Requirements Gathering, Analysis and Specification, Implementation, Testing and Deployment. Where in the Analysis and specification phase a team picks the different frameworks and programming languages, in the Implementation phase these tools have to be used. In this phase the documentation plays an essential role, guiding developers through the execution of the idea. Developers tend to rely on web resources such as documentation, Q\&A sites, as well as tutorials to solve the task at hand. 

There are many reasons why developers rely on web resources aside from the documentation. In a perfect world, the documentation would be complete, maintained, and straight to the point, and would include everything needed to get started and productive. In reality, developers tend to rely on Q\&A sites such as Stack Overflow\footnote{\url{http//stackoverflow.com}} for help when dealing with problems. It is then the task of the developer to collect all of the information found and filter it to find the relevant sections to solve the task at hand. Not only is this process straining, but it can lead to information overload.

To mitigate this phenomenon, summarization can be used as an effective way to automate part of the process, presenting developers only with a subset of the collected data. In this project, we introduce \projectName, a tool which provides the ability to create interactive extractive summaries of a single page or set of pages. By building the tool using HoliRank\cite{Ponz2017a}, an algorithm which builds on the foundations of PageRank\cite{ilprints422}, the heterogeneous nature of artefacts is considered when determining the prominence of a certain section of a document.

To begin with, in section \ref{sec:stateOfTheArt} we take a look at existing approaches for both summarization as well as recommendation systems for Software Engineering. In section \ref{sec:requirementsAndAnalysis} we analyze some of the challenges which we may be faced with, whereas in section \ref{sec:projectDesign} we go in depth an ddiscuss the implementation and design choices to create \projectName. In section \ref{sec:results} we analyze the issues and limitations with our implementation, and in section \ref{sec:implementationIssues} we explore the results.

