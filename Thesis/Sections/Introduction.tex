\section{Introduction}\label{sec:introduction}
To solve the task at hand, developers often rely on web resources. The knowledge that a developer may have is usually not enough, requiring further exploration of documentation, Q\&A sites, and tutorials \cite{A1}. Developers are then faced with a large availability of material, which has to be filtered by either using a few keywords provided by the search engine, or by consulting the results deemed more important. Developers then collect all of the information found and filter it to find the relevant sections needed to solve the task at hand. Not only is this process straining, but it can lead to information overload.

To mitigate this phenomenon, summarization can be used as an effective way to automate part of the process, presenting developers only with a subset of the collected data. Summarization in software engineering is not new, and has been attempted before. Existing approaches include the summarization of email threads \cite{A4} to generate extractive summaries of bug reports as well as exploiting PageRank \cite{ilprints422} to create general summarization approaches for bug reports \cite{A5}. 

The common limitation in the aforementioned approaches is the way artifacts are treated. In both cases, the techniques do not differentiate between purely text-based artifacts from code-based artifacts, but treat both as text-based artifacts. This limits the amount of information that may be extracted from a resource. 

Ponzanelli \textit{et al.} \cite{Ponz2015b} propose a novel approach where the heterogeneous nature of artifacts is considered when performing summarization. The proposed approach augments LexRank \cite{Erkan:2004:LGL:1622487.1622501} to deal with the heteregenous and multidimensional nature of complex artifacts, by providing a new similarity function for heterogeneous entities such as code samples. 

 In this project, we introduce \projectName, a tool which provides the ability to create interactive extractive summaries of a single page or set of pages. By integrating it with LIBRA \cite{Ponz2017a}, a holistic recommender system,  through the usage of HoliRank \cite{Ponz2017a}, an algorithm which builds on the foundations of PageRank\cite{ilprints422}, the heterogeneous nature of artifacts is considered when determining the prominence of a certain section of a document. We use multiple sources, to provide the most complete extractive summary to support developers in filtering and choosing the most relevant parts of different documents.

In Section \ref{sec:stateOfTheArt} we take a look at existing approaches for both summarization as well as recommendation systems for Software Engineering. In Section \ref{sec:requirementsAndAnalysis} we analyze some of the challenges which we may be faced with, whereas in Section \ref{sec:projectDesign} we go in depth and discuss the implementation and design choices to create \projectName. In Section \ref{sec:results} we analyze the issues and limitations with our implementation, and in Section \ref{sec:implementationIssues} we explore the results.

