\label{sec:abstract}
Developers often rely on web resources, such as documentation, Q\&A websites, and tutorials to solve the task at hand. Once the information has been gathered, developers need to manually sort through the collected documents, and find the relevant parts to help them complete the task. Due to the high amount of information available, developers tend to experience information overload, which can be mitigated with the use of summarization. 

%LIBRA\cite{Ponz2017a} is a holistic recommender system that supports the developer in the process of searching and navigating the set of documents returned by a search engine like Google. Although LIBRA guides the user in finding pertinent material, it neither helps with reducing the information overload that may characterize resources that are too long and detailed for the task at hand, nor reduce the information overload occurring when multiple sources have to be aggregated and filtered. 

We introduce \projectName, a summarizer that monitors the pages viewed by a user, extracts the content and creates interactive extractive summaries of a single page or multiple pages, that the user can interact with and decide the amount of filtration she may want to see. By using the algorithm behind LIBRA \cite{Ponz2017a}, a holistic recommender system, we allow the developer to obtain summaries which are pertinent to the previously browsed documents. The project also allows developers to generate a summary of everything that has seen up to a certain moment, giving a high level view of the documents that have been consulted, and once again allowing to extract the most relevant sections to solve the task at hand.