\section{Conclusion}\label{sec:conclusion}
%We presented WebDistiller, a novel approach to reduce information overload which combines a holistic ranking approach, with a simple user interface. Developers have now the ability to partially offload the task of filtering the content, allowing them to focus on the task at hand by consulting extractive summaries of the content being displayed. The summarization is implemented in such a way that we consider the whole history, and not only the current document. We also implemented the ability to generate a summary of the developer's history, allowing them to have a high level overview of the previously seen documents. 

As discussed in Section \ref{sec:introduction}, developers rely on web resources to solve the task at hand. These resources come from a multitude of sources, and it is the task of developers to collect and filter the information available to help them in their task, which can lead to information overload. We believe that summarization is the best approach to help developers, through the automatic filtering of content. We presented \projectName, a novel approach to reduce information overload through the use of extractive interactive summaries. By using HoliRank, the algorithm behind LIBRA, we consider the development context when calculating the prominence of a certain section of a document, as well as considering the hetereogeneous nature of artifacts. 


In Section \ref{sec:requirementsAndAnalysis} we explain the possible challenges of the project, as well as HoliRank, by explaining both LexRank and PageRank from a high level perspective. 
In Section \ref{sec:projectDesign} we started by explaining the theory behind the approach, the basic strategy to perform summarization as well as the context graph, an essential part of this project. We then showed the architecture, and explained both the REST api and the Chrome extension. In Section \ref{sec:results} we explained the tool, showing the functionality as well as giving an example of a summary generated from the development context. Finally in Section \ref{sec:implementationIssues} we discussed possible limitations and improvements to be made to the tool