\section{Conclusion}\label{sec:conclusion}
We presented \projectName, a novel approach to reduce information overload through extractive interactive summaries. We started by defining the possible challenges of the project, and how we may overcome them. We explained how both LexRank and PageRank work, before giving an overview of HoliRank, the algorithm behind LIBRA, which we used to perform summarization on documents. We discussed the general procedure, from the initial stage of data extraction, all the way to filtering. We described our approach more in depth, by explaining the context graph and how the development context may alter the resulting summary, even for a single document. The architecture of the project was next, we explored the different components such as the web service, StORMeD and the Chrome extension. We outlined the procedure of ranking the parts of a document and the criteria for filtering the elements based on a user set threshold. The user interface was presented, and an example of a multi-document summary was shown. Thus, we discussed the possible limitations of the project, and proposed possible solutions to mitigate those issues.
